% -------------------- Packages --------------------
%  Document Class --------------------
\documentclass[12pt,a4paper]{report}
 
% -------------------- Language and Encoding --------------------
\usepackage[utf8]{inputenc}    % Input encoding
\usepackage[T1]{fontenc}       % Font encoding
\usepackage[english]{babel}    % Language

%  Typography and Fonts --------------------
\usepackage{lmodern}           % Improved font rendering

%  Page Layout --------------------
\usepackage[a4paper, margin=1in]{geometry}  % Page size and margins

%  Colors and Graphics --------------------
\usepackage{xcolor}            % Custom colors
\usepackage{graphicx}          % For including images

%  PDF and External Docs --------------------
\usepackage{pdfpages}          % Insert external PDFs
\usepackage{afterpage}         % Delayed page commands

%  Title and Section Formatting --------------------
\usepackage{titlesec}          % Custom section formatting
\usepackage{titling}           % Advanced title control

%  Hyperlinks --------------------
% \usepackage[hidelinks]{hyperref} % Clickable links without colored boxes
\usepackage[draft]{hyperref} % Clickable links without colored boxes

%  Spacing and Paragraphs --------------------
\usepackage{setspace}          % Optional: control line spacing
\usepackage{indentfirst}       % Indent first paragraph after section
\setlength{\parindent}{15pt}   % Paragraph indentation
\setlength{\parskip}{0.7em}    % Space between paragraphs

%  Section Numbering and Depth --------------------
\renewcommand{\thesection}{\Roman{section}}
\renewcommand{\thesubsection}{\thesection.\arabic{subsection}}
\renewcommand{\thesubsubsection}{\thesubsection.\alph{subsubsection}}
\setcounter{tocdepth}{4}       % Table of contents depth

%  Custom Section Styles --------------------
\definecolor{oxfordblue}{RGB}{63,62,136}
\definecolor{lightblue}{RGB}{0,190,213}

\titleformat{\chapter}[hang]{\normalfont\Huge\bfseries\color{black}}{\thechapter}{2pc}{}
\titleformat{\section}[hang]{\normalfont\Large\bfseries\color{oxfordblue}}{\thesection}{1em}{}
\titleformat{\subsection}[hang]{\normalfont\large\bfseries\color{lightblue}}{\thesubsection}{1em}{}
\titleformat{\subsubsection}[hang]{\normalfont\normalsize\bfseries\color{gray}}{\thesubsubsection}{1em}{}

\titleclass{\subsubsubsection}{straight}[\subsubsection]
\newcounter{subsubsubsection}[subsubsection]
\renewcommand\thesubsubsubsection{\thesubsubsection.\roman{subsubsubsection}}
\titleformat{\subsubsubsection}
  {\normalfont\normalsize\bfseries}   % format: normal size, bold
  {\thesubsubsubsection}              % label (the number)
  {1em}                               % separation between number and title
  {}                                  % before-code (empty)
\titlespacing*{\subsubsubsection}
  {0pt}                               % left margin
  {2.5ex plus 1ex minus .2ex}         % before skip (vertical space above)
  {1ex}                               % after skip (vertical space below)
\setcounter{secnumdepth}{4}   % allow numbering down to new level
\setcounter{tocdepth}{4}      % include it in the TOC

%  Abbreviations/Glossaries --------------------
\usepackage[automake]{glossaries-extra}
\makeglossaries
\setabbreviationstyle[acronym]{long-short}

\newcommand{\g}[1]{\gls{#1}}     % Capitalized
\newcommand{\G}[1]{\Gls{#1}}     % Capitalized
\newcommand{\gp}[1]{\glspl{#1}}  % Plural
\newcommand{\Gp}[1]{\Glspl{#1}}  % Capitalized plural

% Simple acronyms {acronym}{short}{long}
\newacronym{mtl}{MTL}{medial temporal lobe}
\newacronym{erp}{ERP}{event-related potential}
\newacronym{emd}{EMD}{empirical mode decomposition}
\newacronym{tmemd}{tmEMD}{tailored masked EMD}
\newacronym{pac}{PAC}{phase-amplitude coupling}
\newacronym{swr}{SWR}{sharp-wave ripple}
\newacronym{lfp}{LFP}{local field potential}
\newacronym{ied}{IED}{interictal epileptiform discharge}
\newacronym{fmri}{fMRI}{functional magnetic resonance imaging}
\newacronym{rem}{REM}{rapid eye movement}
\newacronym{umap}{UMAP}{uniform manifold approximation and projection}
\newacronym{isomap}{ISOMAP}{isometric mapping}

% Manage plurals
\newacronym
  [plural=fast-frequency activities, shortplural=FFAs]
  {ffa}{FFA}{fast-frequency activity}
\newacronym[
    plural=stereoelectroencephalographies,
    shortplural=sEEGs
]{seeg}{sEEG}{stereoelectroencephalography}
\newacronym[
    plural=depth electroencephalographies,
    shortplural=depth EEGs
]{deeg}{depth EEG}{depth electroencephalography}
\newacronym[
    plural=holo-Hilbert spectral analyses,
    shortplural=HHSAs
]{hhsa}{HHSA}{holo-Hilbert spectral analysis}
\newacronym[
    plural=magnetoencephalographies,
    shortplural=MEGs
]{meg}{MEG}{magnetoencephalography}
\newacronym[
    plural=electrocorticographies,
    shortplural=ECoGs
]{ecog}{ECoG}{electrocorticography}


\glsadd{mtl}
\glsadd{deeg}
\glsadd{ffa}
\glsadd{erp}
\glsadd{emd}
\glsadd{tmemd}
\glsadd{pac}
\glsadd{hhsa}
\glsadd{swr}
\glsadd{lfp}
\glsadd{ied}
\glsadd{fmri}
\glsadd{meg}
\glsadd{ecog}
\glsadd{umap}
\glsadd{isomap}
\glsadd{rem}

% ----------------------------- Begin document -----------------------------
\begin{document}

\begin{titlepage}
    \centering
    \vspace*{1.5cm}

    {\Huge\bfseries Investigating Neuronal Network Dynamics Supporting Memory in the Human Brain\\}
    \vspace{2cm}

    \includegraphics[width=0.25\textwidth]{ox_logo.png}
    \vspace{2cm}

    {\Large Thesis\\}
    \vspace{1cm}

    {\Large Adrien A. Causse\\}
    \vspace{0.4cm}
    {\large New College\\}
    {\large University of Oxford\\}
    \vspace{1.5cm}

    {\Large\bfseries Supervisors\\}
    \vspace{0.2cm}
    {\large
        Prof.\ David Dupret\\
        Prof.\ Timothy Denison
    }
    \vspace{1.5cm}

    {\large Trinity Term 2026\\}

    \vfill
\end{titlepage}

% ------------------ Abstract ------------------
\chapter*{Abstract} % Just for info for the assessors
Abstract to write here

% ------------------ TOC ------------------
\tableofcontents

% ------------------ List of abbreviations ------------------
\newpage
\printglossary[type=\acronymtype, title=List of Abbreviations, nonumberlist]

% ------------------ Chapters ------------------
% ----------------------------------- INTRODUCTION ------------------------------------
\chapter*{Introduction} % Two to three pages
\section*{Theta oscillations in animal models}
\subsection*{Rodents}
\subsubsection*{Memory}

% ------------------------------------ Chap 1 ------------------------------------ CHAP 1: Behaviour (PRELIMINARY)
\chapter{Evaluating Memory in Humans}

\section{Associative Memory in Humans}
\subsection{Short-Term and Long-Term Memory}
\subsection{Inference Tasks in Cognitive Psychology}
\subsection{The Role of the Hippocampus in Inference Behaviour}
\subsubsection{Animal Studies}
\subsubsection{Human Lesion Studies}
\subsubsection{Indirect Recordings of Brain Electrical Activity in Humans (\g{fmri}, \g{meg})}
\subsubsection{Direct Recordings of Brain Electrical Activity in Humans}

\section{Investigating Inference using a Social Community Task}
\subsection{Behavioural Paradigm}
\subsection{Variants}
\subsubsection{Simple and Complex Tasks}
\subsubsection{Scientific Rationale for Population Diversity} % sampling epileptic and healthy subjects
\subsubsection{Stimulus Types and Controls}
\subsubsection{Additional Visual Controls}

\section{Quantifying Behavioural Performance}
\subsection{Participant Demographics}
\subsection{Performance Metrics}
\subsubsection{Group-Level Performance}
\subsubsection{Inter-Individual Variability and Performance Profiles}
\subsubsection{Across-Group Comparisons} % compare both populations
\subsection{What Other Factors Explain Performance?}
\subsubsection{Demographic and Cognitive Contributors} % effect of years of study, age, in two pop
\subsubsection{Standardised Cognitive Testing} % neuropsy

\section{Discussion and Conclusion}
\subsection{Inter-Individual Variability in Memory Performance}
\subsection{The Hippocampus Is Central for Inference Performance}
\subsection{Limitations and Considerations}

% ------------------------------------ Chap 2 ------------------------------------ CHAP 2: 2-Hz 
\chapter{Neural activity in the online human hippocampus is paced by a 2-Hz rhythm}

\section{Hippocampal 2-Hz tracks mnemonic engagement}
\subsection{Decomposition of the human hippocampal \g{lfp}}

\subsubsection{Prominent 2-Hz bursts structure hippocampal \g{lfp}} % Mainly a methods section to introduce EMD, bursts, and basic oscillatory analyses
\subsubsubsection{Using \gls{tmemd} to detect slow oscillations}
Description of the usual EMD. Why it fails when not optimised especially in the context of important inter-subject variaiblity. Then tailored masked EMD with optimisation of consistency and mode mixing.
IMF PSDs across contacts => frequency range of the detected oscillations. 
Here also present wavelet spectrograms so the reader can understand how these two methods compare.
How EMD captures non-linearities in the signal (phase-frequency plots) => hippocampal 2-Hz is particularly non-linear.

\subsubsubsection{Hippocampal 2-Hz oscillations are transient}
Detection of IMF cycles. Detection of discrete oscillatory bursts.
Show multiple examples of 2-Hz bursts across contacts and subjects, particularly in contacts clear from IEDs.
Quantification of bursts duration.

\subsubsubsection{Local referencing reduces detection of slow oscillations}
Local referencing on micro and bipolar referencing on macro
=> This is why we will be using CAR throughout the manuscript

\subsubsubsection{Slow-oscillation amplitude and \g{ied} rate}
Detection of IEDs (methods). IEDs are transient, non-oscillatory events. IEDs rate increases at rest. 1- and 2-Hz oscilltaions are more prominent in contacts clear from IEDs. 

\subsubsubsection{Phase reversal of hippocampal 2-Hz oscillations}
Echo to the introduction where we will have presented how the dipole is structured between layers, in humans and rodents. Show maybe one laminar recording from rodents. Then show phase reversal with cycle-triggered average of LFPs.

\subsubsection{Hippocampal 2-Hz is evoked in the memory task}
\subsubsubsection{Hippocampal 2-Hz power increase with task engagement}
Methods: one-over-f fitting. Results: Example contact; estimation plots with various controls; linear mixed-effects models. This is all using contacts free of interictal discharges (reader will unders why because we explained in the previous subsection).

\subsubsubsection{Hippocampal 2-Hz bursts are evoked by mnemonic cues}
ERPs changes throughout the task. Evoked 1-, 2- and 6-Hz amplitudes relate to mnemonic engagement. Correlation between evoked ERPs deflection and 2-Hz ampltitude. 

\section{Hippocampal 2-Hz organizes local neuronal activity}
\subsection{}

\section{Hippocampal 2-Hz synchronizes neuronal activity in the \g{mtl}}
\subsection{}


\subsection{}
\subsubsection{}


% ----------------------------------- DISCUSSION, CONCLUSION, PERSPECTIVES ------------------------------------
\chapter{Discussion}

\end{document}