% -------------------- Packages --------------------
%  Document Class --------------------
\documentclass[12pt,a4paper]{report}
 
% -------------------- Language and Encoding --------------------
\usepackage[utf8]{inputenc}    % Input encoding
\usepackage[T1]{fontenc}       % Font encoding
\usepackage[english]{babel}    % Language

%  Typography and Fonts --------------------
\usepackage{lmodern}           % Improved font rendering

%  Page Layout --------------------
\usepackage[a4paper, margin=1in]{geometry}  % Page size and margins

%  Colors and Graphics --------------------
\usepackage{xcolor}            % Custom colors
\usepackage{graphicx}          % For including images

%  PDF and External Docs --------------------
\usepackage{pdfpages}          % Insert external PDFs
\usepackage{afterpage}         % Delayed page commands

%  Title and Section Formatting --------------------
\usepackage{titlesec}          % Custom section formatting
\usepackage{titling}           % Advanced title control

%  Hyperlinks --------------------
% \usepackage[hidelinks]{hyperref} % Clickable links without colored boxes
\usepackage[draft]{hyperref} % Clickable links without colored boxes

%  Spacing and Paragraphs --------------------
\usepackage{setspace}          % Optional: control line spacing
\usepackage{indentfirst}       % Indent first paragraph after section
\setlength{\parindent}{15pt}   % Paragraph indentation
\setlength{\parskip}{0.7em}    % Space between paragraphs

%  Section Numbering and Depth --------------------
\renewcommand{\thesection}{\Roman{section}}
\renewcommand{\thesubsection}{\thesection.\arabic{subsection}}
\renewcommand{\thesubsubsection}{\thesubsection.\alph{subsubsection}}
\setcounter{tocdepth}{4}       % Table of contents depth

%  Custom Section Styles --------------------
\titleformat{\chapter}[hang]{\normalfont\Huge\bfseries\color{black}}{\thechapter}{2pc}{}
\titleformat{\section}[hang]{\normalfont\Large\bfseries\color{orange}}{\thesection}{1em}{}
\titleformat{\subsection}[hang]{\normalfont\large\bfseries\color{teal}}{\thesubsection}{1em}{}
\titleformat{\subsubsection}[hang]{\normalfont\normalsize\bfseries\color{gray}}{\thesubsubsection}{1em}{}

%  Abbreviations/Glossaries --------------------
\usepackage[automake]{glossaries-extra}
\makeglossaries
\setabbreviationstyle[acronym]{long-short}

\newcommand{\g}[1]{\gls{#1}}     % Capitalized
\newcommand{\G}[1]{\Gls{#1}}     % Capitalized
\newcommand{\gp}[1]{\glspl{#1}}  % Plural
\newcommand{\Gp}[1]{\Glspl{#1}}  % Capitalized plural

% Simple acronyms
\newacronym{mtl}{MTL}{medial temporal lobe}
\newacronym{erp}{ERP}{event-related potential}
\newacronym{emd}{EMD}{empirical mode decomposition}
\newacronym{pac}{PAC}{phase-amplitude coupling}
\newacronym{swr}{SWR}{sharp-wave ripple}
\newacronym{lfp}{LFP}{local field potential}
\newacronym{ied}{IED}{interictal epileptiform discharge}
\newacronym{fmri}{fMRI}{functional magnetic resonance imaging}
\newacronym{rem}{REM}{rapid eye movement}
\newacronym{umap}{UMAP}{uniform manifold approximation and projection}
\newacronym{isomap}{ISOMAP}{isometric mapping}
% Manage plurals
\newacronym
  [plural=fast-frequency activities, shortplural=FFAs]
  {ffa}{FFA}{fast-frequency activity}
\newacronym[
    plural=stereoelectroencephalographies,
    shortplural=sEEGs
]{seeg}{sEEG}{stereoelectroencephalography}
\newacronym[
    plural=depth electroencephalographies,
    shortplural=depth EEGs
]{deeg}{depth EEG}{depth electroencephalography}
\newacronym[
    plural=holo-Hilbert spectral analyses,
    shortplural=HHSAs
]{hhsa}{HHSA}{holo-Hilbert spectral analysis}
\newacronym[
    plural=magnetoencephalographies,
    shortplural=MEGs
]{meg}{MEG}{magnetoencephalography}
\newacronym[
    plural=electrocorticographies,
    shortplural=ECoGs
]{ecog}{ECoG}{electrocorticography}

\glsadd{mtl}
\glsadd{deeg}
\glsadd{ffa}
\glsadd{erp}
\glsadd{emd}
\glsadd{pac}
\glsadd{hhsa}
\glsadd{swr}
\glsadd{lfp}
\glsadd{ied}
\glsadd{fmri}
\glsadd{meg}
\glsadd{ecog}
\glsadd{umap}
\glsadd{isomap}
\glsadd{rem}

% ----------------------------- Begin document -----------------------------
\begin{document}

\begin{titlepage}
    \centering
    \vspace*{1.5cm}

    {\Huge\bfseries Investigating Neuronal Network Dynamics Supporting Memory in the Human Brain\\}
    \vspace{2cm}

    \includegraphics[width=0.25\textwidth]{ox_logo.png}
    \vspace{2cm}

    {\Large Thesis\\}
    \vspace{1cm}

    {\Large Adrien A. Causse\\}
    \vspace{0.4cm}
    {\large New College\\}
    {\large University of Oxford\\}
    \vspace{1.5cm}

    {\Large\bfseries Supervisors\\}
    \vspace{0.2cm}
    {\large
        Prof.\ David Dupret\\
        Prof.\ Timothy Denison
    }
    \vspace{1.5cm}

    {\large Trinity Term 2025\\}

    \vfill
\end{titlepage}

% ------------------ Abstract ------------------
\chapter*{Abstract} % Just for info for the assessors
Abstract to write here

% ------------------ TOC ------------------
\tableofcontents

% ------------------ List of abbreviations ------------------
\newpage
\printglossary[type=\acronymtype, title=List of Abbreviations, nonumberlist]

% ------------------ Chapters ------------------
% ----------------------------------- INTRODUCTION ------------------------------------
\chapter*{Introduction} % Two to three pages

% ------------------------------------ Chap 1 ------------------------------------ ETHICS, RECORDING METHODS, ORIGIN OF LFPs
\chapter{Direct Recordings of Brain Activity with Hybrid Electrodes in Humans}

\section{Experimental Rationale and Design}
\subsection{Motivation for Direct Human Recordings}
\subsubsection{The Value of \G{deeg} for Systems Neuroscience} % historical milestones
\subsubsection{Unique Opportunities and Challenges of Human Data} % ethics, disease context
\subsection{Technical Foundations}
\subsubsection{The Use of Hybrid Electrodes}
\subsubsection{Comparisons with Animal and Indirect Recordings in Humans}

\section{Electrodes and Recording Methods}
\subsection{Overview of Available Techniques}
\subsubsection{Cortical Grids and \g{ecog}}
\subsubsection{Intra-Operative Recordings}
\subsubsection{Behnke-Fried Hybrid Electrodes}
\subsubsection{DIXI MICRODEEP Hybrid Electrodes} % Deep and lateral tetrodes
\subsection{Localization of \g{deeg} Contacts}
\subsubsection{Imaging Pipeline, Segmentation and Coregistration}
\subsubsection{Hippocampal Subfield Segmentation}
\subsubsection{Estimating Contact Locations Using Electrode Models}
\subsubsection{Manual Verification}
\subsection{The Impact of Referencing Schemes}
\subsubsection{Monopolar Referencing}
\subsubsection{Bipolar Referencing}
\subsubsection{Common Average Referencing}
\subsection{Relation Between \gp{lfp} and Spiking Activity}
\subsubsection{The Influence of Contact Type, Distance and Reference}
\subsubsection{\g{ffa} Is a Proxy of Local Neuronal Firing} % describe how you get it

\section{Physiological Variability Across Conditions}
\subsection{Sources of Background Neural Activity} % Definition and Detection of \gp{ied} and other interictal entities
\subsection{Contextual Modulation of Background Neural Activity} % IEDs rate
\subsubsection{Cognitive Engagement}
\subsubsection{Sleep Stages}
\subsubsection{Anatomical and Functional Gradients}

\section{Discussion and Conclusion}

% ------------------------------------ Chap 2 ------------------------------------ BEHAVIOUR
\chapter{Evaluating Memory and Inference Performance in Humans}

\section{Associative Memory in Humans}
\subsection{Short-Term and Long-Term Memory}
\subsection{Inference Tasks in Cognitive Psychology}
\subsection{The Role of the Hippocampus in Inference Behaviour}
\subsubsection{Animal Studies}
\subsubsection{Human Lesion Studies}
\subsubsection{Indirect Recordings of Brain Electrical Activity in Humans (\g{fmri}, \g{meg})}
\subsubsection{Direct Recordings of Brain Electrical Activity in Humans}

\section{Investigating Inference using a Social Community Task}
\subsection{Behavioural Paradigm}
\subsection{Variants}
\subsubsection{Simple and Complex Tasks}
\subsubsection{Scientific Rationale for Population Diversity} % sampling epileptic and healthy subjects
\subsubsection{Stimulus Types and Controls}
\subsubsection{Additional Visual Controls}

\section{Quantifying Behavioural Performance}
\subsection{Participant Demographics}
\subsection{Performance Metrics}
\subsubsection{Group-Level Performance}
\subsubsection{Inter-Individual Variability and Performance Profiles}
\subsubsection{Across-Group Comparisons} % comapre both populations
\subsection{What Other Factors Explain Performance?}
\subsubsection{Demographic and Cognitive Contributors} % effect of years of study, age, in two pop
\subsubsection{Standardised Cognitive Testing} % neuropsy

\section{Hippocampal Activity Correlates with Task Performance}
\subsection{Learning-Induced Changes in Visual Response in the Hippocampus}
\subsubsection{Evoked-Related Potentials in Screening Sessions}
\subsubsection{\g{ffa} in Screening Sessions}
\subsection{Hippocampal \g{ffa} Predicts Task Performance}
\subsubsection{Learning performance}
\subsubsection{Memory performance}
\subsubsection{Inference performance}

\section{Discussion and Conclusion}
\subsection{Inter-Individual Variability in Memory Performance}
\subsection{The Hippocampus Is Central for Inference Performance}
\subsection{Limitations and Considerations}

% ------------------------------------ Chap 3 ------------------------------------ SLOW OSCILLATIONS IN THE MTL
\chapter{Slow Oscillations Organize Neuronal Activity in the \g{mtl}}

\section{Why Study Oscillations in the \g{mtl}?}
\subsection{Network Implications of Oscillations}
\subsubsection{Post-Synaptic Potentials}
\subsubsection{Coordinating Neuronal Activity}
\subsubsection{"Chunking" Information on a Cycle-by-Cycle Basis}
\subsection{Conservation of \g{mtl} Structures Across Species}
\subsubsection{Developmental and Genetical Conservation of the \g{mtl}}
\subsubsection{Histological Conservation of the \g{mtl}}
\subsubsection{Anatomical Conservation of the \g{mtl}}
\subsection{Theta Oscillations in Non-Human Studies}
\subsubsection{Type I and Type II Theta}
\subsubsection{The Active Sensing Hypothesis Across Species}
\subsubsection{Theta Oscillations Across Brain Regions}
\subsubsection{\g{rem} sleep in Non-Human Studies}
\subsection{Slow Oscillations in Humans}
\subsubsection{A Fluid Definition of "Theta" oscillations in Human Literature}
\subsubsection{Memory Tasks}
\subsubsection{Virtual and Real-World Navigation}
\subsubsection{\g{rem} sleep in Human Studies}

\section{Analysing Oscillatory Components Using \g{emd}}
\subsection{Description of \g{emd}}
\subsubsection{\g{emd} and ensemble \g{emd}}
\subsubsection{Masked \g{emd}}
\subsection{Cycle-by-Cycle Analysis}
\subsubsection{Detecting Oscillatory Cycles}
\subsubsection{Oscillatory Bout Detection}
\subsection{Measuring Amplitude Modulation and Phase Locking}
\subsubsection{\g{hhsa}}
\subsubsection{\g{pac} and Phase Randomization}
\subsubsection{Spike-to-Phase Locking}

\section{Behavioural-State Dependence of Hippocampal 2Hz}
\subsection{Hippocampal 2Hz in \g{rem} Sleep}
\subsubsection{Polysomnography and Sleep Scoring}
\subsubsection{2Hz Power in \g{rem} Sleep}
\subsubsection{Neuronal Activity Locks to Hippocampal 2Hz in \g{rem} Sleep}
\subsubsection{Asymmetry of Hippocampal 2Hz in Wake and \g{rem} Sleep}
\subsection{Hippocampal 2Hz Increases During Task Engagement}
\subsubsection{Hippocampal 2Hz Power Between Task Stages}
\subsubsection{Hippocampal 2Hz is Evoked by Stimulus Onset}

\section{2Hz Oscillations Across the \g{mtl}}
\subsection{Anatomical Gradients of 2Hz Oscillations in the \g{mtl}}
\subsubsection{Power}
\subsubsection{Amplitude Modulation}
\subsubsection{\g{pac}}
\subsubsection{Coupling of Neuronal Activity to Local Oscillations}
\subsection{Hippocampal 2Hz Coordinates Activity in the Entire \g{mtl}}
\subsubsection{Neurons of the \g{mtl} Show 2Hz Rhythmicity}
\subsubsection{Coupling of \g{ffa} to Hippocampal 2Hz Phase}
\subsubsection{Coupling of Neuronal Activity to Hippocampal 2Hz Phase}

\section{Discussion and Conclusion}
\subsection{The Role of 2Hz Oscillations in Human Cognition}
\subsection{Scaling Hypotheses and Structural Constraints}

% ------------------------------------ Chap 4 ------------------------------------ SINGLE-UNIT ACTIVITY
\chapter{Characterisation of Single-neuron Activity in the \g{mtl}}

\section{Accessing Neuronal Diversity Using Electrophysiological Properties}
\subsection{How to Differentiate Neuronal Subtypes Based on Electrophysiological Properties?}
\subsubsection{Pyramidal Neurons and Interneurons}
\subsubsection{Pyramidal Neurons Are Diverse}
\subsubsection{Interneurons Are Diverse}
\subsection{The Single-Neuron Code}
\subsubsection{Concepts cells}
\subsubsection{Memory-Guided Behaviour}
\subsection{The Populational Code}
\subsubsection{Cell Assemblies and Coactivity}
\subsubsection{Attempts to Investigate the Populational Code in Humans}
\subsubsection{Limits of In-Vivo Recordings In Humans}

\section{Methods}
\subsection{Spike detection, Unit Isolation and Quality Control}
\subsubsection{Artefact Detection}
\subsubsection{Semi-Automated Spike Sorting}
\subsubsection{Post-Hoc Quality Control}
\subsection{Extraction of Electrophysiological Features}
\subsubsection{Waveform Features}
\subsubsection{Firing Features}
\subsubsection{Pyramidal-Like and Interneuron-Like Profiles} % validation with cross correlograms
\subsubsection{Coupling to Oscillations}
\subsection{Identification of Responsive Units}
\subsubsection{Peri-Stimulus Time Histograms}
\subsubsection{Classification Methods}

\section{Properties of Temporal Lobe Neurons}
\subsection{General Properties Across Regions}
\subsubsection{Waveform Features}
\subsubsection{Firing Features}
\subsection{Controlling for Differences Across Anatomical and Functional Sites} % Higher in Affected Tissue
\subsubsection{Baseline Firing Rate} % affected VS unaffected zones
\subsubsection{\gp{ied} and Fast-Ripples}
\subsection{Coupling to Slow Oscillations Differentiates Two Neuronal Populations}
\subsubsection{Baseline Properties}
\subsubsection{Evoked 2Hz}
\subsubsection{Engagement in \gp{swr}} % This is still quite hypothetical 

\section{Single-Neuron Activity in Relation to The Inference Task}
\subsection{Response Profiles to Visual Cues Across Regions}
\subsubsection{Unspecific Responses} % Better describe HPC reponse profiles
\subsubsection{Specific Responses} % Concept cells and fusiform gyrus cells
\subsubsection{Reponse Delays} % Especially useful to characterize FG cells which may lock to different parts of the ERPs
\subsection{Can \g{mtl} Single-Neuron Activity Inform Behaviour?}
\subsubsection{Representation Similarity Analyses} % Pre-screening VS Post-Screening 
\subsubsection{Predicting Learning and Test Performance using Single-Neuron Activity}

\section{Discussion and Conclusion}
\subsection{Neuronal Sub-Populations}
\subsection{Challenges and Future Directions}

% ----------------------------------- Chap 5 ------------------------------------ SHARP-WAVE RIPPLES
\chapter{Hippocampal \gp{swr} and Reactivation in Humans}
\section{A Brief Introduction to \gp{swr}}
\subsection{\gp{swr} in Non-Human Studies}
\subsubsection{\gp{swr} Physiology}
\subsubsection{Downstream Readers}
\subsubsection{The Complex Relation Between Reactivation and \gp{swr}}
\subsection{\gp{swr} and Reactivation in Human Studies}
\subsubsection{A Fluid Definition of \gp{swr} in Human Literature}
\subsubsection{Comparison With Rodent Studies}
\subsubsection{Reactivation in Humans}

\section{Detection and Definition of \gp{swr}}
\subsection{Initial \gp{swr} detection}
\subsubsection{Selection of Contacts} % Reference contact in the white matter and contacts in CA1/SUB/PRESUB
\subsubsection{Ripple Band Filtering and Threshold Crossing}
\subsubsection{Exclusion of \gp{ied}}
\subsection{Validation Step}
\subsubsection{Number of Ripple Cycles}
\subsubsection{\g{umap} and \g{isomap}}

\section{Description of \gp{swr} Basic Properties}
\subsection{Basic Internal Properties}
\subsubsection{Frequency and Amplitude Gradients}
\subsubsection{Micro-Macro Correspondence} % Using macro sharp wave to infer tetrode position 
\subsection{Behavioural State Dependence}
\subsubsection{Slow-Wave Sleep} % Comparison of IEDs and SWRs / between different slow wave sleep stages
\subsubsection{Rest Periods in Task}

\section{Reactivation and \gp{swr}}
\subsection{Coordination of Brain-Wide Activation Patterns During \gp{swr}}
\subsubsection{Hierarchy of FFA During \gp{swr}}
\subsubsection{Reactivation of Brain-Wide FFA in \gp{swr}} % 2Hz VS 6Hz / Task
\subsubsection{Reactivation of Brain-Wide FFA without \gp{swr}}
\subsection{Coordination of Neuronal Activity During \gp{swr}}
\subsubsection{\g{mtl} Neuronal Activity During \gp{swr}}
\subsubsection{Reactivation of Neuronal Activity in \gp{swr}}
\subsubsection{Reactivation of Neuronal Activity without \gp{swr}}

\section{Discussion}
\subsection{Similarities and Differences With Rodent \gp{swr}}
\subsection{Reactivation and Formation of Long-term Memory}
\section{Conclusion}

% ----------------------------------- DISCUSSION, CONCLUSION, PERSPECTIVES ------------------------------------
\chapter{Discussion}

\section{Integration Across Chapters}

\section{Key Contributions to Neuroscience}
\subsection{Design and Implementation of a Novel Inference Task in Humans}
\subsection{Homology Between Human 2Hz and Rodent Theta Oscillations}
\subsection{Human Neuronal Populations Have Diverse Electrophysiological Properties In-Vivo}
\subsection{Conservation of \gp{swr} Properties Across Species}

\section{Methodological Strengths and Limitations}
\subsection{Strengths}
\subsubsection{Ecological Validity and Multisite Data Acquisition} % Diverse Socio-Economical Backgrounds Represented, Cohort with Heterogeneous Etiologies
\subsubsection{Multi-Scale Recordings and Analytical Approaches}
\subsubsection{Data-Driven Signal Characterization}
\subsubsection{Single-Units are Likely Single-Neurons}

\subsection{Limitations}
\subsection{Controlling for Inter-Subject Variability in Performance} % Participants Suffer From Cognitive Deficits
\subsection{Limitations in Defining Physiological Baseline Activity} % I mean "unaffected" regions are likely not
\subsection{Subregion-Specific and Variable Brain Coverage Across Participants}
\subsection{The Relationship Between \g{ffa} and Local Neuronal Activity Remains Obscure} 
\subsubsection{Why Did We Never Refer to "Gamma" Oscillations?} % Locality
\subsubsection{How Many Neurons Do Macro-Contacts Record From?}

\chapter{Conclusion and Perspectives}
\section{Open Questions and Future Directions}
\section{Potential Applications}

\end{document}